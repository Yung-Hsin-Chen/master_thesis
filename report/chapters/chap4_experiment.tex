\newchap{Experiment}
\label{chap:4_experiment}
%%%%%%%%%%%%%%%%%%%%%%%%%%%%%%%%%%%%%%%%%%%%%%%%%%%%%%%%%%%%%%%%%%%%%%%%%%%%%%%%%%%%%%%%%%%%%%%%
%%                              Data Collection and processing                                %%
%%%%%%%%%%%%%%%%%%%%%%%%%%%%%%%%%%%%%%%%%%%%%%%%%%%%%%%%%%%%%%%%%%%%%%%%%%%%%%%%%%%%%%%%%%%%%%%%
\section{Data Collection and Processing}
\label{sec:3_data_collection_and_processing}
This section is dedicated to detailing the data collection process for OCR task training and CharBERT$_{\text{SMALL}}$ training. We will outline the types and sources of data harnessed for this study, emphasizing the diversity and volume of the datasets to ensure comprehensive learning. Following the data collection overview, we will delve into the processing techniques applied to the collected data.
\subsection{Data for OCR} 
\label{subsec:3_data_for_ocr}
We focus on the performances of handwritten datasets on the composite model. The data used in this study is George Washington (GW) handwritten dataset and IAM handwritten dataset. They serve as a valuable benchmark for developing and evaluating handwriting recognition systems. Note that to ensure comparability with existing studies, this research adheres to the established train-validation-test splits of the GW and IAM datasets. 
\TODO{Add data downloading link}

The GW dataset is a collection of historical letters and dairies handwritten by George Washington and this secretaries in 1755. The dataset is often used in research focused on recognising historical handwriting, which poses unique challenges due to the use of archaic words and phrases, and the degradation of materials over time.

The IAM dataset is a more comtemporary collection of English handwriting samples. It contains forms written by hundreds of writers, and is thus renowned for its diversity in handwriting styles, the variability of written content. 

While the George Washington Handwritten Dataset provides a niche focus on historical documents, making it ideal for projects related to historical document analysis and preservation, the IAM Handwriting Database offers a broad spectrum of modern handwriting samples, making it suitable for a wide range of handwriting recognition applications. Both datasets have played crucial roles in advancing the field of handwriting recognition. 
\FIG{Add data sample here}
\paragraph*{Transcription Ground Truth Processing}
\label{par:3_transcription_gound_truth_processing}
The transcription ground truths of the IAM dataset are stored in XML files, where parsing these files is sufficient to retrieve the transcription texts. On the other hand, the transcription ground truths of the GW dataset follow a more complex format. In this dataset, individual characters within words are separated by hyphens ("-"), and words themselves are separated by vertical bars ("\textbar"). Additionally, punctuation marks are represented by special characters, with a specific table detailing the replacements for each punctuation mark. The table for the punctuation replacement can be found in this \href{https://github.com/Yung-Hsin-Chen/master_thesis/blob/src/model/config/punctuation_list.json}{link}. 
\TODO{Add text example}
For data processing purposes, we modify the transcription texts from the GW dataset by removing the hyphens, replacing vertical bars and special characters representing punctuations with spaces and their respective punctuation marks. This process ensures that the transcription texts are standardized and easily readable. 
\subsection{Data for Training CharBERT$_{\text{SMALL}}$}
\label{subsec:3_data_for_training_charbert}
In the original study, the authors trained CharBERT using a 12GB dataset from Wikipedia. This training process spanned 5 days, utilizing the computational power of two NVIDIA Tesla V100 GPUs. Given constraints in time and computational resources, our approach involves testing a scaled-down version of CharBERT, which we have designated as CharBERT$_{\text{SMALL}}$. This variant was trained on a significantly smaller dataset, specifically 1.13GB of English Wikipedia data, from which sentences were randomly sampled. This adaptation allows us to evaluate the performance of CharBERT under more restricted conditions, ensuring our experiments are feasible within our available resources.
%%%%%%%%%%%%%%%%%%%%%%%%%%%%%%%%%%%%%%%%%%%%%%%%%%%%%%%%%%%%%%%%%%%%%%%%%%%%%%%%%%%%%%%%%%%%%%%%
%%                                       Baseline Model                                       %%
%%%%%%%%%%%%%%%%%%%%%%%%%%%%%%%%%%%%%%%%%%%%%%%%%%%%%%%%%%%%%%%%%%%%%%%%%%%%%%%%%%%%%%%%%%%%%%%%
\section{Baseline Model}
\label{sec:4_baseline_model}
In this study, we consider both the pre-trained and fine-tuned versions of the TrOCR model as baseline models for comparison. The rationale behind using the fine-tuned TrOCR model alongside its pre-trained counterpart is to ascertain the performance benchmark. If the fine-tuned TrOCR model outperform the composite model, it would suggest that further fine-tuning of the TrOCR model is a more efficient approach than employing the larger, more complex composite model. This comparison allows us to evaluate the efficiency of the fine-tuning process relative to the integration of additional model components.
\paragraph*{Pre-trained TrOCR}
\label{par:4_pre-trained_trocr}
We use the pre-trained \href{https://huggingface.co/microsoft/trocr-large-handwritten}{handwritten large TrOCR} as the baseline model. The model is trained on 
\paragraph*{Fine-tuned TrOCR}
\label{par:4_fine_tuned_trocr}
%%%%%%%%%%%%%%%%%%%%%%%%%%%%%%%%%%%%%%%%%%%%%%%%%%%%%%%%%%%%%%%%%%%%%%%%%%%%%%%%%%%%%%%%%%%%%%%%
%%                           Model Training and Evaluation Criteria                           %%
%%%%%%%%%%%%%%%%%%%%%%%%%%%%%%%%%%%%%%%%%%%%%%%%%%%%%%%%%%%%%%%%%%%%%%%%%%%%%%%%%%%%%%%%%%%%%%%%
\section{Composite Model Training and Evaluation Criteria}
\label{sec:3_model_training_and_evaluation_criteria}
In this section, we will be exploring the training and evaluation process of the composite model. Details including optimiser, loss function, hyperparameters, and the challenges encountered training deep models and how to mitigate it will also be discussed.

In this section, we delve into the training and evaluation process for the composite model. The section will cover key aspects such as the choice of optimizer, the loss function employed, and the hyperparameters set for the training. Additionally, the challenges commonly faced when training deep learning models, and insights into strategies and solutions to mitigate these issues will also be discussed.
\subsection{Training Details}
\label{subsec:3_training_details}
% optimiser, loss function, hyperparameters, utilisation of GPU resources
When training deep learning models, the selection of optimisation, loss computation techniques, and the hyperparameters settings play a crucial role in the effeiciency and effectiveness of the training process. In this study, we use Adam as the optimiser and cross-entropy for loss computation.

\paragraph*{Optimiser}
\label{par:3_Optimiser}
Utilising Adam as the optimiser provides a sophisticated approach by adopting an adaptive learning rate, which is particularly adept at managing sparse gradients tasks such as NLP problems. Besides, due to its efficient computation of adaptive learning rates, Adam often leads to faster convergence on training data. This can significantly reduce the time and computational resources needed to train deep models, making the process more efficient.

For the training of the composite model, the learning rate has been meticulously set to $1\mathrm{e}{-5}$. This settingn ensures a controlled and gradual adaptation of the model parameters, facilitating a smooth convergence towards the local minima. Additionally, a weight decay parameter of $1\mathrm{e}{-5}$ is employed to enhance the model's ability to navigate the optimization landscape efficiently. This careful calibration prevents the optimizer from overshooting the local minima, thereby promoting stability in the training process and improving the model's overall performance.

\paragraph*{Loss Function}
\label{par:3_loss_function}
OCR tasks involves predicting the probability distribution of possible characters for a given input image. It is a multi-class classification problem, with each character representing a unique class. Cross-entropy loss is naturally suited for multi-class settings. This makes it directly applicable to the task of classifying images into characters. In addition, unlike other loss functions that might focus solely on the accuracy of the classification, cross-entropy loss encourages the model not just to predict the correct class but to do so with high confidence. High-confidence wrong predictions are penalised more, encouraging the model to be cautious about making predictions when unsure, which is often the case with less frequent characters.
\subsection{Training Challenges and Solutions}
\label{subsec:3_training_challenges_and_solutions}
% overfitting (dropout), underfitting, learning rate, gradient vanishing/exploding, converge faster (norm), transfer learning
Training deep learning models involves navigating a series of common challenges that can significantly impact their performance and effectiveness. Common challenges include overfitting, vanishing/exploding gradients, high computational costs, and data quantity. The following discussion will delve into these challenges and the strategies used in this study to mitigate them.

\paragraph*{Overfitting}
\label{par:3_overfitting}
Overfitting occurs when a model learns the training data too well, capturing noisein the training set instead of learning the underlying patterns, which results in poor generalisation to new, unseen data. Deep learning models, by their very nature, have a large number of parameters, allowing them to model intricate patterns and relationships in the data. However, it also means they have the capacity to memorise irrelevant details in the training data, leading to overfitting. Lack of regularisation techniques, or poorly chosen learning rate and batch size can easily lead to overfitting. 

To mitigate overfitting, we implemented dropout layers between both linear and convolutional layers within the composite model architecture. Dropout serves as a form of regularization that, by temporarily dropping out units from the network, prevents the model from becoming overly dependent on any single element of the training data, thereby enhancing its generalization capabilities.

Additionally, we explored the impact of batch size on model training. Smaller batch sizes result in more noise during the gradient updates, which can have a regularizing effect. However, very small batch sizes can lead to extremely noisy gradient estimates, which might make training unstable or lead to convergence on suboptimal solutions. Through iterative testing, we determined that a batch size of 8 strikes an optimal balance, offering sufficient regularization to mitigate overfitting while maintaining stable and effective training dynamics.

\paragraph*{Vanishing/Exploding Gradients}
\label{par:3_vanishing_exploding_gradients}
Training deep models often encounters exploding or vanishing gradient problems due to their complex architectures and the long chains of computations involved. If the gradients are large (greater than 1), they can exponentially increase as they propagate back through the layers, leading to exploding gradients. Conversely, if the gradients are small (less than 1), they can exponentially decrease, leading to vanishing gradients. 

Certain activation functions, like the sigmoid or tanh, squish a large input space into a small output range in a non-linear fashion. For inputs with large magnitudes, the gradients can be extremely small, leading to vanishing gradients. In addition, improper initialization of weights can exacerbate the exploding or vanishing gradient problems. For instance, large initial weights can lead to exploding gradients, while small initial weights can contribute to vanishing gradients.

To mitigate the vanishing/exploding gradients, we deploy strategies such as Xavier initialisation, Leaky ReLU activation function, gradient clipping, residual connection and batch normalisation in the composite model architecture. The details of residual connection and batch normalisation\CHECK{Make sure they are mentioned} is discussed in \hyperref[subsec:3_composite_model]{Composite Model}.

\paragraph*{High Computational Costs}
\label{par:3_high_computational_costs}
Deep models, especially those with many layers and parameters, require significant computational resources and time to train. Hardware accelerators like GPUs can reduce training times. In this study, a single A100 GPU, equipped with 80GB of RAM, was utilized to accelerate the training process. When applied to the GW dataset and the IAM dataset, the training durations were approximately 2 minutes per epoch and 30 minutes per epoch, respectively.

\paragraph*{Data Quantity}
\label{par:3_data_quantity}
Deep models typically require large datasets to effectively learn and generalize due to their complex architectures and the vast number of parameters they contain. Training these models from scratch on limited data often leads to overfitting. To mitigate this issue, we employ transfer learning. In this study, we take pre-trained CharBERT and TrOCR model, which are developed on large and comprehensive datasets, and adapt it to our specific task. 

\subsection{Evaluation}
\label{subsec:3_evaluation}
The validation set serves the purpose of hyperparameter tuning and model selection. Meanwhile, the testing set is reserved for the final evaluation. For the GW dataset, we employ a 4-fold cross-validation approach. This method divides the dataset into four equally sized segments, using each in turn for testing while the remaining three serve as the training set. The final results for the GW dataset represent the average performance across these four folds.

In this study, we focus on word error rate (WER) and character error rate (CER) as our primary evaluation metrics. These metrics are critical for assessing the model's accuracy in recognizing and reproducing text, providing the model's performance in terms of both word-level and character-level precision.
%%%%%%%%%%%%%%%%%%%%%%%%%%%%%%%%%%%%%%%%%%%%%%%%%%%%%%%%%%%%%%%%%%%%%%%%%%%%%%%%%%%%%%%%%%%%%%%%
%%                          CharBERT Training and Evaluation Criteria                         %%
%%%%%%%%%%%%%%%%%%%%%%%%%%%%%%%%%%%%%%%%%%%%%%%%%%%%%%%%%%%%%%%%%%%%%%%%%%%%%%%%%%%%%%%%%%%%%%%%
\section{CharBERT$_{\text{SMALL}}$ Training and Evaluation Criteria}
\label{sec:3_charbert_training_and_evaluation_criteria}
The training and evaluation of CharBERT$_{\text{SMALL}}$ adhere closely to the methodology outlined by the original authors. The model is trained using the pre-training objectives, i.e., MLM and NLM, over the large text corpus. The training involves backpropagation and optimisation of weights to minimize the prediction error for both MLM and NLM tasks. While adhering to the \href{https://github.com/mawentao277/CharBERT/blob/main/shell/mlm.sh}{hyperparameters recommended by the original authors}, we have adjusted the batch size to 16, up from the suggested size of 4, to accelerate the computation process. The training was executed on five A100 GPUs, each equipped with 80GB of RAM, and was completed over the course of six days, encompassing five training epochs. 

\TODO{Finish this}The results is evaluated
%%%%%%%%%%%%%%%%%%%%%%%%%%%%%%%%%%%%%%%%%%%%%%%%%%%%%%%%%%%%%%%%%%%%%%%%%%%%%%%%%%%%%%%%%%%%%%%%
%%                                       Ablation Study                                       %%
%%%%%%%%%%%%%%%%%%%%%%%%%%%%%%%%%%%%%%%%%%%%%%%%%%%%%%%%%%%%%%%%%%%%%%%%%%%%%%%%%%%%%%%%%%%%%%%%
\section{Ablation Study}
\label{sec:4_ablation_study}
%%%%%%%%%%%%%%%%%%%%%%%%%%%%%%%%%%%%%%%%%%%%%%%%%%%%%%%%%%%%%%%%%%%%%%%%%%%%%%%%%%%%%%%%%%%%%%%%
%%                                          Analysis                                          %%
%%%%%%%%%%%%%%%%%%%%%%%%%%%%%%%%%%%%%%%%%%%%%%%%%%%%%%%%%%%%%%%%%%%%%%%%%%%%%%%%%%%%%%%%%%%%%%%%
\section{Analysis}
